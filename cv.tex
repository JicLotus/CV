

\documentclass[letterpaper]{article}

\usepackage{hyperref}
\usepackage{geometry}
\usepackage{xspace}

\usepackage[T1]{fontenc}
\usepackage[sc,osf]{mathpazo}
\usepackage{amsmath}

\def\name{Jose Ignacio Castelli}
\def\footerlink{https://github.com/JicLotus/CV/CV.pdf}

% PDF metadata
\hypersetup{
  colorlinks = true,
  urlcolor = black,
  pdfauthor = {\name},
  pdfkeywords = {android, software development, algorithms, computer science, mathematics},
  pdftitle = {\name: Curriculum Vitae},
  pdfsubject = {Curriculum Vitae},
  pdfpagemode = UseNone
}

\geometry{
  body={6.5in, 8.5in},
  left=1.0in,
  top=1.25in
}

% Page headers
\pagestyle{myheadings}
\markright{\name}
\thispagestyle{empty}

% Custom section fonts
\usepackage{sectsty}
\sectionfont{\rmfamily\mdseries\Large}
\subsectionfont{\rmfamily\mdseries\itshape\large}

% Don't indent paragraphs.
\setlength\parindent{0em}

% Make lists without bullets
\renewenvironment{itemize}{
  \begin{list}{}{
    \setlength{\leftmargin}{1.5em}
  }
}{
  \end{list}
}

\newenvironment{no-indent-itemize}{
  \begin{list}{}{
    \setlength{\leftmargin}{0em}
  }
}{
  \end{list}
}

\def\tilde{$\scriptstyle\sim$}
\def\bullet{$\circ$\xspace}

\begin{document}

{\huge \name}

\bigskip
\begin{minipage}{0.45\linewidth}
  \begin{tabular}{llll}
    Year of birth: 07-18-2017\\
    Mobile: & +54 11 3697 9500
    \\
    Email: & \href{mailto:joseignaciocastelli92@gmail.com}{\tt joseignaciocastelli92@gmail.com} 
       & Github: &\href{http://github.com/jiclotus}{\tt http://github.com/jiclotus}\\
    Languages: & \textsc{en}, \textsc{es}
  \end{tabular}
\end{minipage}

\bigskip
\textsc{Main Programming Languages}: C\#,C++

\section*{Summary}
\begin{no-indent-itemize}
    \item Since I was thirteen years old I have been developing video games and company management systems in different languages. 
\end{no-indent-itemize}

\section*{Education}
\begin{no-indent-itemize}
  \item  Software Engineering from University of Buenos Aires 2011-2016 ( It is a 6-year university course equivalent to Bachelor's and master's degree) 
\end{no-indent-itemize}


\section*{Employment}
\begin{no-indent-itemize}

\item
    \textsc{Freelance .Net Developer at Supervielle Bank | Jan 2017 - Present}
    \begin{itemize}
    \item\bullet Billing management system in C\# and .Net framework to bank entity.
    \end{itemize}
    
\item \textsc{C\# QR App septiembre de 2015 - mayo de 2016}
\begin{itemize} 
\item\bullet 
This project is for the control of coils of a recycling plant. 
It was developed in C\#, Php and Android. 
Source \& Documentation: 
https://github.com/JicLotus/Control-Sistematico-QR
\end{itemize}


\item \textsc{C++ Game Development diciembre de 2011 - julio de 2014 In 2012 }
\begin{itemize} \item\bullet
This game was online with 167 players simultaneously. Now the game status is offline; however, it is under development. https://www.facebook.com/InmortalAO/
\end{itemize}

\item \textsc{Visual Basic 6.0 Game Development at NRG Games enero de 2008 - }
\begin{itemize} \item\bullet
Dec 2008  Second experience with another game Sponsor(NRG Games) in the development of Argentum Online 2D project. 
\end{itemize}


\item \textsc{Visual Basic 6.0 Game Development at LocalStrike Jan 2007 - Dec 2007}
\begin{itemize} \item\bullet
    I had a game sponsor called LocalStrike for my Argentum Online 2D Game Project.
\end{itemize}


\end{no-indent-itemize}


\section*{Projects}
\begin{no-indent-itemize}
  \item \textsc{3D WebGL Graphic Escene} | May 2016 - Jun 2016
    \begin{itemize}
    \item\bullet 3D graphic scene developed in WebGL and JS.
    \item\bullet Each vertex point in the graphic scene was positioned mathematically
    \end{itemize}
    \begin{itemize}
    \item Source \& Documentation: \href{https://github.com/JicLotus/3DGraphicEscene}{https://github.com/JicLotus/3DGraphicEscene}
    
    \end{itemize}

\item \textsc{C++/Android Dropbox Open Source} | Jul 2015 - Jul 2015
    \begin{itemize}
        \item\bullet Project description: I developed a Dropbox open source for Android. It has the same options as Dropbox. The web server was developed in RocksDB in C++ language.
        
        Source \& Documentation: \href{https://github.com/JicLotus/Dropbox-source}{https://github.com/JicLotus/Dropbox-source}
    \end{itemize}

\item \textsc{Capacitive touch sensors} |  Jul 2015 - Jul 2015

    \begin{itemize}
        \item\bullet Project description This project consists in the implementation of two capacitive touch sensors using a Atmega88pa microcontroller. These sensors were used for the control of intensity of a 12V light.
        
        Source \& Documentation: \href{https://github.com/JicLotus/Capacitive-Sensor}{https://github.com/JicLotus/Capacitive-Sensor}
    \end{itemize}

\end{no-indent-itemize}

\bigskip
\begin{center}
  \begin{footnotesize}
    Last updated: \today \\
    \href{\footerlink}{\texttt{\footerlink}}
  \end{footnotesize}
\end{center}

\end{document}